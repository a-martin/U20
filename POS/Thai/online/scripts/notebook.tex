
% Default to the notebook output style

    


% Inherit from the specified cell style.




    
\documentclass[11pt]{article}

    
    
    \usepackage[T1]{fontenc}
    % Nicer default font (+ math font) than Computer Modern for most use cases
    \usepackage{mathpazo}

    % Basic figure setup, for now with no caption control since it's done
    % automatically by Pandoc (which extracts ![](path) syntax from Markdown).
    \usepackage{graphicx}
    % We will generate all images so they have a width \maxwidth. This means
    % that they will get their normal width if they fit onto the page, but
    % are scaled down if they would overflow the margins.
    \makeatletter
    \def\maxwidth{\ifdim\Gin@nat@width>\linewidth\linewidth
    \else\Gin@nat@width\fi}
    \makeatother
    \let\Oldincludegraphics\includegraphics
    % Set max figure width to be 80% of text width, for now hardcoded.
    \renewcommand{\includegraphics}[1]{\Oldincludegraphics[width=.8\maxwidth]{#1}}
    % Ensure that by default, figures have no caption (until we provide a
    % proper Figure object with a Caption API and a way to capture that
    % in the conversion process - todo).
    \usepackage{caption}
    \DeclareCaptionLabelFormat{nolabel}{}
    \captionsetup{labelformat=nolabel}

    \usepackage{adjustbox} % Used to constrain images to a maximum size 
    \usepackage{xcolor} % Allow colors to be defined
    \usepackage{enumerate} % Needed for markdown enumerations to work
    \usepackage{geometry} % Used to adjust the document margins
    \usepackage{amsmath} % Equations
    \usepackage{amssymb} % Equations
    \usepackage{textcomp} % defines textquotesingle
    % Hack from http://tex.stackexchange.com/a/47451/13684:
    \AtBeginDocument{%
        \def\PYZsq{\textquotesingle}% Upright quotes in Pygmentized code
    }
    \usepackage{upquote} % Upright quotes for verbatim code
    \usepackage{eurosym} % defines \euro
    \usepackage[mathletters]{ucs} % Extended unicode (utf-8) support
    \usepackage[utf8x]{inputenc} % Allow utf-8 characters in the tex document
    \usepackage{fancyvrb} % verbatim replacement that allows latex
    \usepackage{grffile} % extends the file name processing of package graphics 
                         % to support a larger range 
    % The hyperref package gives us a pdf with properly built
    % internal navigation ('pdf bookmarks' for the table of contents,
    % internal cross-reference links, web links for URLs, etc.)
    \usepackage{hyperref}
    \usepackage{longtable} % longtable support required by pandoc >1.10
    \usepackage{booktabs}  % table support for pandoc > 1.12.2
    \usepackage[inline]{enumitem} % IRkernel/repr support (it uses the enumerate* environment)
    \usepackage[normalem]{ulem} % ulem is needed to support strikethroughs (\sout)
                                % normalem makes italics be italics, not underlines
    

    
    
    % Colors for the hyperref package
    \definecolor{urlcolor}{rgb}{0,.145,.698}
    \definecolor{linkcolor}{rgb}{.71,0.21,0.01}
    \definecolor{citecolor}{rgb}{.12,.54,.11}

    % ANSI colors
    \definecolor{ansi-black}{HTML}{3E424D}
    \definecolor{ansi-black-intense}{HTML}{282C36}
    \definecolor{ansi-red}{HTML}{E75C58}
    \definecolor{ansi-red-intense}{HTML}{B22B31}
    \definecolor{ansi-green}{HTML}{00A250}
    \definecolor{ansi-green-intense}{HTML}{007427}
    \definecolor{ansi-yellow}{HTML}{DDB62B}
    \definecolor{ansi-yellow-intense}{HTML}{B27D12}
    \definecolor{ansi-blue}{HTML}{208FFB}
    \definecolor{ansi-blue-intense}{HTML}{0065CA}
    \definecolor{ansi-magenta}{HTML}{D160C4}
    \definecolor{ansi-magenta-intense}{HTML}{A03196}
    \definecolor{ansi-cyan}{HTML}{60C6C8}
    \definecolor{ansi-cyan-intense}{HTML}{258F8F}
    \definecolor{ansi-white}{HTML}{C5C1B4}
    \definecolor{ansi-white-intense}{HTML}{A1A6B2}

    % commands and environments needed by pandoc snippets
    % extracted from the output of `pandoc -s`
    \providecommand{\tightlist}{%
      \setlength{\itemsep}{0pt}\setlength{\parskip}{0pt}}
    \DefineVerbatimEnvironment{Highlighting}{Verbatim}{commandchars=\\\{\}}
    % Add ',fontsize=\small' for more characters per line
    \newenvironment{Shaded}{}{}
    \newcommand{\KeywordTok}[1]{\textcolor[rgb]{0.00,0.44,0.13}{\textbf{{#1}}}}
    \newcommand{\DataTypeTok}[1]{\textcolor[rgb]{0.56,0.13,0.00}{{#1}}}
    \newcommand{\DecValTok}[1]{\textcolor[rgb]{0.25,0.63,0.44}{{#1}}}
    \newcommand{\BaseNTok}[1]{\textcolor[rgb]{0.25,0.63,0.44}{{#1}}}
    \newcommand{\FloatTok}[1]{\textcolor[rgb]{0.25,0.63,0.44}{{#1}}}
    \newcommand{\CharTok}[1]{\textcolor[rgb]{0.25,0.44,0.63}{{#1}}}
    \newcommand{\StringTok}[1]{\textcolor[rgb]{0.25,0.44,0.63}{{#1}}}
    \newcommand{\CommentTok}[1]{\textcolor[rgb]{0.38,0.63,0.69}{\textit{{#1}}}}
    \newcommand{\OtherTok}[1]{\textcolor[rgb]{0.00,0.44,0.13}{{#1}}}
    \newcommand{\AlertTok}[1]{\textcolor[rgb]{1.00,0.00,0.00}{\textbf{{#1}}}}
    \newcommand{\FunctionTok}[1]{\textcolor[rgb]{0.02,0.16,0.49}{{#1}}}
    \newcommand{\RegionMarkerTok}[1]{{#1}}
    \newcommand{\ErrorTok}[1]{\textcolor[rgb]{1.00,0.00,0.00}{\textbf{{#1}}}}
    \newcommand{\NormalTok}[1]{{#1}}
    
    % Additional commands for more recent versions of Pandoc
    \newcommand{\ConstantTok}[1]{\textcolor[rgb]{0.53,0.00,0.00}{{#1}}}
    \newcommand{\SpecialCharTok}[1]{\textcolor[rgb]{0.25,0.44,0.63}{{#1}}}
    \newcommand{\VerbatimStringTok}[1]{\textcolor[rgb]{0.25,0.44,0.63}{{#1}}}
    \newcommand{\SpecialStringTok}[1]{\textcolor[rgb]{0.73,0.40,0.53}{{#1}}}
    \newcommand{\ImportTok}[1]{{#1}}
    \newcommand{\DocumentationTok}[1]{\textcolor[rgb]{0.73,0.13,0.13}{\textit{{#1}}}}
    \newcommand{\AnnotationTok}[1]{\textcolor[rgb]{0.38,0.63,0.69}{\textbf{\textit{{#1}}}}}
    \newcommand{\CommentVarTok}[1]{\textcolor[rgb]{0.38,0.63,0.69}{\textbf{\textit{{#1}}}}}
    \newcommand{\VariableTok}[1]{\textcolor[rgb]{0.10,0.09,0.49}{{#1}}}
    \newcommand{\ControlFlowTok}[1]{\textcolor[rgb]{0.00,0.44,0.13}{\textbf{{#1}}}}
    \newcommand{\OperatorTok}[1]{\textcolor[rgb]{0.40,0.40,0.40}{{#1}}}
    \newcommand{\BuiltInTok}[1]{{#1}}
    \newcommand{\ExtensionTok}[1]{{#1}}
    \newcommand{\PreprocessorTok}[1]{\textcolor[rgb]{0.74,0.48,0.00}{{#1}}}
    \newcommand{\AttributeTok}[1]{\textcolor[rgb]{0.49,0.56,0.16}{{#1}}}
    \newcommand{\InformationTok}[1]{\textcolor[rgb]{0.38,0.63,0.69}{\textbf{\textit{{#1}}}}}
    \newcommand{\WarningTok}[1]{\textcolor[rgb]{0.38,0.63,0.69}{\textbf{\textit{{#1}}}}}
    
    
    % Define a nice break command that doesn't care if a line doesn't already
    % exist.
    \def\br{\hspace*{\fill} \\* }
    % Math Jax compatability definitions
    \def\gt{>}
    \def\lt{<}
    % Document parameters
    \title{Analysis of U20 Thai data}
    
    
    

    % Pygments definitions
    
\makeatletter
\def\PY@reset{\let\PY@it=\relax \let\PY@bf=\relax%
    \let\PY@ul=\relax \let\PY@tc=\relax%
    \let\PY@bc=\relax \let\PY@ff=\relax}
\def\PY@tok#1{\csname PY@tok@#1\endcsname}
\def\PY@toks#1+{\ifx\relax#1\empty\else%
    \PY@tok{#1}\expandafter\PY@toks\fi}
\def\PY@do#1{\PY@bc{\PY@tc{\PY@ul{%
    \PY@it{\PY@bf{\PY@ff{#1}}}}}}}
\def\PY#1#2{\PY@reset\PY@toks#1+\relax+\PY@do{#2}}

\expandafter\def\csname PY@tok@gd\endcsname{\def\PY@tc##1{\textcolor[rgb]{0.63,0.00,0.00}{##1}}}
\expandafter\def\csname PY@tok@gu\endcsname{\let\PY@bf=\textbf\def\PY@tc##1{\textcolor[rgb]{0.50,0.00,0.50}{##1}}}
\expandafter\def\csname PY@tok@gt\endcsname{\def\PY@tc##1{\textcolor[rgb]{0.00,0.27,0.87}{##1}}}
\expandafter\def\csname PY@tok@gs\endcsname{\let\PY@bf=\textbf}
\expandafter\def\csname PY@tok@gr\endcsname{\def\PY@tc##1{\textcolor[rgb]{1.00,0.00,0.00}{##1}}}
\expandafter\def\csname PY@tok@cm\endcsname{\let\PY@it=\textit\def\PY@tc##1{\textcolor[rgb]{0.25,0.50,0.50}{##1}}}
\expandafter\def\csname PY@tok@vg\endcsname{\def\PY@tc##1{\textcolor[rgb]{0.10,0.09,0.49}{##1}}}
\expandafter\def\csname PY@tok@vi\endcsname{\def\PY@tc##1{\textcolor[rgb]{0.10,0.09,0.49}{##1}}}
\expandafter\def\csname PY@tok@vm\endcsname{\def\PY@tc##1{\textcolor[rgb]{0.10,0.09,0.49}{##1}}}
\expandafter\def\csname PY@tok@mh\endcsname{\def\PY@tc##1{\textcolor[rgb]{0.40,0.40,0.40}{##1}}}
\expandafter\def\csname PY@tok@cs\endcsname{\let\PY@it=\textit\def\PY@tc##1{\textcolor[rgb]{0.25,0.50,0.50}{##1}}}
\expandafter\def\csname PY@tok@ge\endcsname{\let\PY@it=\textit}
\expandafter\def\csname PY@tok@vc\endcsname{\def\PY@tc##1{\textcolor[rgb]{0.10,0.09,0.49}{##1}}}
\expandafter\def\csname PY@tok@il\endcsname{\def\PY@tc##1{\textcolor[rgb]{0.40,0.40,0.40}{##1}}}
\expandafter\def\csname PY@tok@go\endcsname{\def\PY@tc##1{\textcolor[rgb]{0.53,0.53,0.53}{##1}}}
\expandafter\def\csname PY@tok@cp\endcsname{\def\PY@tc##1{\textcolor[rgb]{0.74,0.48,0.00}{##1}}}
\expandafter\def\csname PY@tok@gi\endcsname{\def\PY@tc##1{\textcolor[rgb]{0.00,0.63,0.00}{##1}}}
\expandafter\def\csname PY@tok@gh\endcsname{\let\PY@bf=\textbf\def\PY@tc##1{\textcolor[rgb]{0.00,0.00,0.50}{##1}}}
\expandafter\def\csname PY@tok@ni\endcsname{\let\PY@bf=\textbf\def\PY@tc##1{\textcolor[rgb]{0.60,0.60,0.60}{##1}}}
\expandafter\def\csname PY@tok@nl\endcsname{\def\PY@tc##1{\textcolor[rgb]{0.63,0.63,0.00}{##1}}}
\expandafter\def\csname PY@tok@nn\endcsname{\let\PY@bf=\textbf\def\PY@tc##1{\textcolor[rgb]{0.00,0.00,1.00}{##1}}}
\expandafter\def\csname PY@tok@no\endcsname{\def\PY@tc##1{\textcolor[rgb]{0.53,0.00,0.00}{##1}}}
\expandafter\def\csname PY@tok@na\endcsname{\def\PY@tc##1{\textcolor[rgb]{0.49,0.56,0.16}{##1}}}
\expandafter\def\csname PY@tok@nb\endcsname{\def\PY@tc##1{\textcolor[rgb]{0.00,0.50,0.00}{##1}}}
\expandafter\def\csname PY@tok@nc\endcsname{\let\PY@bf=\textbf\def\PY@tc##1{\textcolor[rgb]{0.00,0.00,1.00}{##1}}}
\expandafter\def\csname PY@tok@nd\endcsname{\def\PY@tc##1{\textcolor[rgb]{0.67,0.13,1.00}{##1}}}
\expandafter\def\csname PY@tok@ne\endcsname{\let\PY@bf=\textbf\def\PY@tc##1{\textcolor[rgb]{0.82,0.25,0.23}{##1}}}
\expandafter\def\csname PY@tok@nf\endcsname{\def\PY@tc##1{\textcolor[rgb]{0.00,0.00,1.00}{##1}}}
\expandafter\def\csname PY@tok@si\endcsname{\let\PY@bf=\textbf\def\PY@tc##1{\textcolor[rgb]{0.73,0.40,0.53}{##1}}}
\expandafter\def\csname PY@tok@s2\endcsname{\def\PY@tc##1{\textcolor[rgb]{0.73,0.13,0.13}{##1}}}
\expandafter\def\csname PY@tok@nt\endcsname{\let\PY@bf=\textbf\def\PY@tc##1{\textcolor[rgb]{0.00,0.50,0.00}{##1}}}
\expandafter\def\csname PY@tok@nv\endcsname{\def\PY@tc##1{\textcolor[rgb]{0.10,0.09,0.49}{##1}}}
\expandafter\def\csname PY@tok@s1\endcsname{\def\PY@tc##1{\textcolor[rgb]{0.73,0.13,0.13}{##1}}}
\expandafter\def\csname PY@tok@dl\endcsname{\def\PY@tc##1{\textcolor[rgb]{0.73,0.13,0.13}{##1}}}
\expandafter\def\csname PY@tok@ch\endcsname{\let\PY@it=\textit\def\PY@tc##1{\textcolor[rgb]{0.25,0.50,0.50}{##1}}}
\expandafter\def\csname PY@tok@m\endcsname{\def\PY@tc##1{\textcolor[rgb]{0.40,0.40,0.40}{##1}}}
\expandafter\def\csname PY@tok@gp\endcsname{\let\PY@bf=\textbf\def\PY@tc##1{\textcolor[rgb]{0.00,0.00,0.50}{##1}}}
\expandafter\def\csname PY@tok@sh\endcsname{\def\PY@tc##1{\textcolor[rgb]{0.73,0.13,0.13}{##1}}}
\expandafter\def\csname PY@tok@ow\endcsname{\let\PY@bf=\textbf\def\PY@tc##1{\textcolor[rgb]{0.67,0.13,1.00}{##1}}}
\expandafter\def\csname PY@tok@sx\endcsname{\def\PY@tc##1{\textcolor[rgb]{0.00,0.50,0.00}{##1}}}
\expandafter\def\csname PY@tok@bp\endcsname{\def\PY@tc##1{\textcolor[rgb]{0.00,0.50,0.00}{##1}}}
\expandafter\def\csname PY@tok@c1\endcsname{\let\PY@it=\textit\def\PY@tc##1{\textcolor[rgb]{0.25,0.50,0.50}{##1}}}
\expandafter\def\csname PY@tok@fm\endcsname{\def\PY@tc##1{\textcolor[rgb]{0.00,0.00,1.00}{##1}}}
\expandafter\def\csname PY@tok@o\endcsname{\def\PY@tc##1{\textcolor[rgb]{0.40,0.40,0.40}{##1}}}
\expandafter\def\csname PY@tok@kc\endcsname{\let\PY@bf=\textbf\def\PY@tc##1{\textcolor[rgb]{0.00,0.50,0.00}{##1}}}
\expandafter\def\csname PY@tok@c\endcsname{\let\PY@it=\textit\def\PY@tc##1{\textcolor[rgb]{0.25,0.50,0.50}{##1}}}
\expandafter\def\csname PY@tok@mf\endcsname{\def\PY@tc##1{\textcolor[rgb]{0.40,0.40,0.40}{##1}}}
\expandafter\def\csname PY@tok@err\endcsname{\def\PY@bc##1{\setlength{\fboxsep}{0pt}\fcolorbox[rgb]{1.00,0.00,0.00}{1,1,1}{\strut ##1}}}
\expandafter\def\csname PY@tok@mb\endcsname{\def\PY@tc##1{\textcolor[rgb]{0.40,0.40,0.40}{##1}}}
\expandafter\def\csname PY@tok@ss\endcsname{\def\PY@tc##1{\textcolor[rgb]{0.10,0.09,0.49}{##1}}}
\expandafter\def\csname PY@tok@sr\endcsname{\def\PY@tc##1{\textcolor[rgb]{0.73,0.40,0.53}{##1}}}
\expandafter\def\csname PY@tok@mo\endcsname{\def\PY@tc##1{\textcolor[rgb]{0.40,0.40,0.40}{##1}}}
\expandafter\def\csname PY@tok@kd\endcsname{\let\PY@bf=\textbf\def\PY@tc##1{\textcolor[rgb]{0.00,0.50,0.00}{##1}}}
\expandafter\def\csname PY@tok@mi\endcsname{\def\PY@tc##1{\textcolor[rgb]{0.40,0.40,0.40}{##1}}}
\expandafter\def\csname PY@tok@kn\endcsname{\let\PY@bf=\textbf\def\PY@tc##1{\textcolor[rgb]{0.00,0.50,0.00}{##1}}}
\expandafter\def\csname PY@tok@cpf\endcsname{\let\PY@it=\textit\def\PY@tc##1{\textcolor[rgb]{0.25,0.50,0.50}{##1}}}
\expandafter\def\csname PY@tok@kr\endcsname{\let\PY@bf=\textbf\def\PY@tc##1{\textcolor[rgb]{0.00,0.50,0.00}{##1}}}
\expandafter\def\csname PY@tok@s\endcsname{\def\PY@tc##1{\textcolor[rgb]{0.73,0.13,0.13}{##1}}}
\expandafter\def\csname PY@tok@kp\endcsname{\def\PY@tc##1{\textcolor[rgb]{0.00,0.50,0.00}{##1}}}
\expandafter\def\csname PY@tok@w\endcsname{\def\PY@tc##1{\textcolor[rgb]{0.73,0.73,0.73}{##1}}}
\expandafter\def\csname PY@tok@kt\endcsname{\def\PY@tc##1{\textcolor[rgb]{0.69,0.00,0.25}{##1}}}
\expandafter\def\csname PY@tok@sc\endcsname{\def\PY@tc##1{\textcolor[rgb]{0.73,0.13,0.13}{##1}}}
\expandafter\def\csname PY@tok@sb\endcsname{\def\PY@tc##1{\textcolor[rgb]{0.73,0.13,0.13}{##1}}}
\expandafter\def\csname PY@tok@sa\endcsname{\def\PY@tc##1{\textcolor[rgb]{0.73,0.13,0.13}{##1}}}
\expandafter\def\csname PY@tok@k\endcsname{\let\PY@bf=\textbf\def\PY@tc##1{\textcolor[rgb]{0.00,0.50,0.00}{##1}}}
\expandafter\def\csname PY@tok@se\endcsname{\let\PY@bf=\textbf\def\PY@tc##1{\textcolor[rgb]{0.73,0.40,0.13}{##1}}}
\expandafter\def\csname PY@tok@sd\endcsname{\let\PY@it=\textit\def\PY@tc##1{\textcolor[rgb]{0.73,0.13,0.13}{##1}}}

\def\PYZbs{\char`\\}
\def\PYZus{\char`\_}
\def\PYZob{\char`\{}
\def\PYZcb{\char`\}}
\def\PYZca{\char`\^}
\def\PYZam{\char`\&}
\def\PYZlt{\char`\<}
\def\PYZgt{\char`\>}
\def\PYZsh{\char`\#}
\def\PYZpc{\char`\%}
\def\PYZdl{\char`\$}
\def\PYZhy{\char`\-}
\def\PYZsq{\char`\'}
\def\PYZdq{\char`\"}
\def\PYZti{\char`\~}
% for compatibility with earlier versions
\def\PYZat{@}
\def\PYZlb{[}
\def\PYZrb{]}
\makeatother


    % Exact colors from NB
    \definecolor{incolor}{rgb}{0.0, 0.0, 0.5}
    \definecolor{outcolor}{rgb}{0.545, 0.0, 0.0}



    
    % Prevent overflowing lines due to hard-to-break entities
    \sloppy 
    % Setup hyperref package
    \hypersetup{
      breaklinks=true,  % so long urls are correctly broken across lines
      colorlinks=true,
      urlcolor=urlcolor,
      linkcolor=linkcolor,
      citecolor=citecolor,
      }
    % Slightly bigger margins than the latex defaults
    
    \geometry{verbose,tmargin=1in,bmargin=1in,lmargin=1in,rmargin=1in}
    
    

    \begin{document}
    
    
    \maketitle
    
    

    
    \section{Analysis of U20 Thai data}\label{analysis-of-u20-thai-data}

    \subsection{Quick summary}\label{quick-summary}

Thai is an isomorphic, harmonic language, with the following NP
structure:

N Adj Num Dem

It is the mirror image of English.

Participants in this experiment were exposed to an English-like version
of Thai where modifiers consistently preceded the noun they modified.

Note that Thai normally uses classifiers that were omitted in this
experiment. This makes it less ecological than the English equivalent
experiment (Culbertson \& Adger, 2014).

    \subsection{Data checking and basic
plotting}\label{data-checking-and-basic-plotting}

Load libraries and data in.

    \begin{Verbatim}[commandchars=\\\{\}]
{\color{incolor}In [{\color{incolor}2}]:} \PY{c+c1}{\PYZsh{} allow inline plotting}
        \PY{o}{\PYZpc{}}\PY{k}{matplotlib} inline
        
        \PY{c+c1}{\PYZsh{} dataframe package}
        \PY{k+kn}{import} \PY{n+nn}{pandas} \PY{k+kn}{as} \PY{n+nn}{pd}
        \PY{c+c1}{\PYZsh{} matplotlib wrapper for easier plotting}
        \PY{k+kn}{import} \PY{n+nn}{seaborn} \PY{k+kn}{as} \PY{n+nn}{sns}
        \PY{c+c1}{\PYZsh{} allow math functions}
        \PY{k+kn}{from} \PY{n+nn}{numpy} \PY{k+kn}{import} \PY{o}{*} 
        
        \PY{c+c1}{\PYZsh{} \PYZpc{}run clean\PYZhy{}data.py \PYZsh{} concatenates and cleans data into DataFrame called df}
        \PY{n}{df} \PY{o}{=} \PY{n}{pd}\PY{o}{.}\PY{n}{read\PYZus{}csv}\PY{p}{(}\PY{l+s+s1}{\PYZsq{}}\PY{l+s+s1}{../data/U20Thai\PYZus{}TestData.csv}\PY{l+s+s1}{\PYZsq{}}\PY{p}{)}
\end{Verbatim}


    Check that data were loaded correctly.

    \begin{Verbatim}[commandchars=\\\{\}]
{\color{incolor}In [{\color{incolor}3}]:} \PY{n}{df}\PY{o}{.}\PY{n}{sample}\PY{p}{(}\PY{l+m+mi}{5}\PY{p}{)}
\end{Verbatim}


\begin{Verbatim}[commandchars=\\\{\}]
{\color{outcolor}Out[{\color{outcolor}3}]:}                                                      ID     sex  \textbackslash{}
        3966  id\_Wed Apr 19 2017 13:34:06 GMT+0700 (SE Asia {\ldots}  female   
        832   id\_Fri Apr 28 2017 12:13:05 GMT+0700 (SE Asia {\ldots}    male   
        2770  id\_Wed Apr 19 2017 10:22:24 GMT+0700 (SE Asia {\ldots}  female   
        236   id\_Fri Apr 21 2017 10:49:25 GMT+0700 (SE Asia {\ldots}  female   
        3879  id\_Wed Apr 19 2017 13:29:48 GMT+0700 (SE Asia {\ldots}    male   
        
                     condition                                             trials  \textbackslash{}
        3966  condition\_AdjNum  testTrial15\_inner-outer\_1\_four blue vase\_blue {\ldots}   
        832   condition\_NumDem  testTrial1\_outer\_2\_car these\_these car\_----\_--{\ldots}   
        2770  condition\_AdjDem  testTrial19\_inner-outer\_4\_tray those blue\_blue{\ldots}   
        236   condition\_NumDem  testTrial45\_inner\_2\_tray four\_four tray\_----\_-{\ldots}   
        3879  condition\_AdjNum  testTrial8\_inner-outer\_4\_couch two furry\_furry{\ldots}   
        
             phase        trial         type  correctChoice          choice1  \textbackslash{}
        3966  test  testTrial15  inner-outer              1   four blue vase   
        832   test   testTrial1        outer              2        car these   
        2770  test  testTrial19  inner-outer              4  tray those blue   
        236   test  testTrial45        inner              2        tray four   
        3879  test   testTrial8  inner-outer              4  couch two furry   
        
                      choice2      {\ldots}         mod oldNoun oldMod1 oldMod2  \textbackslash{}
        3966   blue four vase      {\ldots}        vase    True    True    True   
        832         these car      {\ldots}         car    True    True     NaN   
        2770  blue those tray      {\ldots}        tray    True    True    True   
        236         four tray      {\ldots}        tray   False    True     NaN   
        3879  furry two couch      {\ldots}       couch    True    True    True   
        
                choiceType1    choiceType2    choiceType3    choiceType4 choice  \textbackslash{}
        3966  Outer-Inner-N  Inner-Outer-N  N-Outer-Inner  N-Inner-Outer      2   
        832             NaN            NaN            NaN            NaN      2   
        2770  N-Outer-Inner  Inner-Outer-N  N-Inner-Outer  Outer-Inner-N      4   
        236             NaN            NaN            NaN            NaN      2   
        3879  N-Outer-Inner  Inner-Outer-N  N-Inner-Outer  Outer-Inner-N      2   
        
             binaryCorrect  
        3966             0  
        832              1  
        2770             1  
        236              1  
        3879             0  
        
        [5 rows x 23 columns]
\end{Verbatim}
            
    Select data from the test phase only.

    \begin{Verbatim}[commandchars=\\\{\}]
{\color{incolor}In [{\color{incolor}4}]:} \PY{n}{df} \PY{o}{=} \PY{n}{df}\PY{p}{[}\PY{n}{df}\PY{o}{.}\PY{n}{phase}\PY{o}{==}\PY{l+s+s1}{\PYZsq{}}\PY{l+s+s1}{test}\PY{l+s+s1}{\PYZsq{}}\PY{p}{]}
\end{Verbatim}


    Check how many participants we have in each condition:

    \begin{Verbatim}[commandchars=\\\{\}]
{\color{incolor}In [{\color{incolor}5}]:} \PY{n}{df}\PY{o}{.}\PY{n}{groupby}\PY{p}{(}\PY{l+s+s1}{\PYZsq{}}\PY{l+s+s1}{condition}\PY{l+s+s1}{\PYZsq{}}\PY{p}{)}\PY{o}{.}\PY{n}{ID}\PY{o}{.}\PY{n}{nunique}\PY{p}{(}\PY{p}{)}
\end{Verbatim}


\begin{Verbatim}[commandchars=\\\{\}]
{\color{outcolor}Out[{\color{outcolor}5}]:} condition
        condition\_AdjDem    20
        condition\_AdjNum    21
        condition\_NumDem    16
        Name: ID, dtype: int64
\end{Verbatim}
            
    Check by-participant accuracy on single-modifier trials. The critical
column is \texttt{post} denoting whether or not (1 or 0) a response was
postnominal.

    \begin{Verbatim}[commandchars=\\\{\}]
{\color{incolor}In [{\color{incolor}6}]:} \PY{n}{singleMods} \PY{o}{=} \PY{n}{df}\PY{p}{[}\PY{n}{df}\PY{o}{.}\PY{n}{type}\PY{o}{.}\PY{n}{isin}\PY{p}{(}\PY{p}{[}\PY{l+s+s1}{\PYZsq{}}\PY{l+s+s1}{inner}\PY{l+s+s1}{\PYZsq{}}\PY{p}{,}\PY{l+s+s1}{\PYZsq{}}\PY{l+s+s1}{outer}\PY{l+s+s1}{\PYZsq{}}\PY{p}{]}\PY{p}{)}\PY{p}{]}\PY{o}{.}\PY{n}{groupby}\PY{p}{(}\PY{p}{[}\PY{l+s+s1}{\PYZsq{}}\PY{l+s+s1}{ID}\PY{l+s+s1}{\PYZsq{}}\PY{p}{,} \PY{l+s+s1}{\PYZsq{}}\PY{l+s+s1}{condition}\PY{l+s+s1}{\PYZsq{}}\PY{p}{]}\PY{p}{,} \PY{n}{as\PYZus{}index}\PY{o}{=}\PY{n+nb+bp}{False}\PY{p}{)}\PY{o}{.}\PY{n}{binaryCorrect}\PY{o}{.}\PY{n}{mean}\PY{p}{(}\PY{p}{)}
\end{Verbatim}


    Plot histogram of single-modifier trial accuracy.

    \begin{Verbatim}[commandchars=\\\{\}]
{\color{incolor}In [{\color{incolor}7}]:} \PY{n}{ax} \PY{o}{=} \PY{n}{df}\PY{p}{[}\PY{n}{df}\PY{o}{.}\PY{n}{type}\PY{o}{.}\PY{n}{isin}\PY{p}{(}\PY{p}{[}\PY{l+s+s1}{\PYZsq{}}\PY{l+s+s1}{inner}\PY{l+s+s1}{\PYZsq{}}\PY{p}{,}\PY{l+s+s1}{\PYZsq{}}\PY{l+s+s1}{outer}\PY{l+s+s1}{\PYZsq{}}\PY{p}{]}\PY{p}{)}\PY{p}{]}\PY{o}{.}\PY{n}{groupby}\PY{p}{(}\PY{p}{[}\PY{l+s+s1}{\PYZsq{}}\PY{l+s+s1}{ID}\PY{l+s+s1}{\PYZsq{}}\PY{p}{,} \PY{l+s+s1}{\PYZsq{}}\PY{l+s+s1}{condition}\PY{l+s+s1}{\PYZsq{}}\PY{p}{]}\PY{p}{)}\PY{o}{.}\PY{n}{binaryCorrect}\PY{o}{.}\PY{n}{mean}\PY{p}{(}\PY{p}{)}\PY{o}{.}\PY{n}{hist}\PY{p}{(}\PY{n}{color}\PY{o}{=}\PY{l+s+s1}{\PYZsq{}}\PY{l+s+s1}{green}\PY{l+s+s1}{\PYZsq{}}\PY{p}{)}
        \PY{n}{ax}\PY{o}{.}\PY{n}{set\PYZus{}xlim}\PY{p}{(}\PY{p}{(}\PY{l+m+mi}{0}\PY{p}{,}\PY{l+m+mi}{1}\PY{p}{)}\PY{p}{)}
\end{Verbatim}


\begin{Verbatim}[commandchars=\\\{\}]
{\color{outcolor}Out[{\color{outcolor}7}]:} (0, 1)
\end{Verbatim}
            
    \begin{center}
    \adjustimage{max size={0.9\linewidth}{0.9\paperheight}}{output_13_1.png}
    \end{center}
    { \hspace*{\fill} \\}
    
    Participants are clearly responding correctly on single modifier trials,
with only a couple of uncooperative people below the 0.85 threshold.
Their data are excluded.

    \begin{Verbatim}[commandchars=\\\{\}]
{\color{incolor}In [{\color{incolor}8}]:} \PY{n}{badParticipants} \PY{o}{=} \PY{n}{singleMods}\PY{p}{[}\PY{n}{singleMods}\PY{o}{.}\PY{n}{binaryCorrect}\PY{o}{\PYZlt{}}\PY{l+m+mf}{0.85}\PY{p}{]}\PY{o}{.}\PY{n}{ID}\PY{o}{.}\PY{n}{unique}\PY{p}{(}\PY{p}{)}
        \PY{n}{df} \PY{o}{=} \PY{n}{df}\PY{p}{[}\PY{o}{\PYZti{}}\PY{n}{df}\PY{o}{.}\PY{n}{ID}\PY{o}{.}\PY{n}{isin}\PY{p}{(}\PY{n}{badParticipants}\PY{p}{)}\PY{p}{]}
\end{Verbatim}


    Now check N's again to see how many participants remain after exclusion.

    \begin{Verbatim}[commandchars=\\\{\}]
{\color{incolor}In [{\color{incolor}9}]:} \PY{n}{df}\PY{o}{.}\PY{n}{groupby}\PY{p}{(}\PY{l+s+s1}{\PYZsq{}}\PY{l+s+s1}{condition}\PY{l+s+s1}{\PYZsq{}}\PY{p}{)}\PY{o}{.}\PY{n}{ID}\PY{o}{.}\PY{n}{nunique}\PY{p}{(}\PY{p}{)}
\end{Verbatim}


\begin{Verbatim}[commandchars=\\\{\}]
{\color{outcolor}Out[{\color{outcolor}9}]:} condition
        condition\_AdjDem    18
        condition\_AdjNum    18
        condition\_NumDem    15
        Name: ID, dtype: int64
\end{Verbatim}
            
    Plot histogram of two-modifier trial accuracy for remaining
participants.

    \begin{Verbatim}[commandchars=\\\{\}]
{\color{incolor}In [{\color{incolor}10}]:} \PY{n}{ax} \PY{o}{=} \PY{n}{df}\PY{p}{[}\PY{n}{df}\PY{o}{.}\PY{n}{type}\PY{o}{==}\PY{l+s+s1}{\PYZsq{}}\PY{l+s+s1}{inner\PYZhy{}outer}\PY{l+s+s1}{\PYZsq{}}\PY{p}{]}\PY{o}{.}\PY{n}{groupby}\PY{p}{(}\PY{p}{[}\PY{l+s+s1}{\PYZsq{}}\PY{l+s+s1}{ID}\PY{l+s+s1}{\PYZsq{}}\PY{p}{,} \PY{l+s+s1}{\PYZsq{}}\PY{l+s+s1}{condition}\PY{l+s+s1}{\PYZsq{}}\PY{p}{]}\PY{p}{)}\PY{o}{.}\PY{n}{binaryCorrect}\PY{o}{.}\PY{n}{mean}\PY{p}{(}\PY{p}{)}\PY{o}{.}\PY{n}{hist}\PY{p}{(}\PY{n}{color}\PY{o}{=}\PY{l+s+s1}{\PYZsq{}}\PY{l+s+s1}{green}\PY{l+s+s1}{\PYZsq{}}\PY{p}{)}
         \PY{n}{ax}\PY{o}{.}\PY{n}{set\PYZus{}xlim}\PY{p}{(}\PY{p}{(}\PY{l+m+mi}{0}\PY{p}{,}\PY{l+m+mi}{1}\PY{p}{)}\PY{p}{)}
\end{Verbatim}


\begin{Verbatim}[commandchars=\\\{\}]
{\color{outcolor}Out[{\color{outcolor}10}]:} (0, 1)
\end{Verbatim}
            
    \begin{center}
    \adjustimage{max size={0.9\linewidth}{0.9\paperheight}}{output_19_1.png}
    \end{center}
    { \hspace*{\fill} \\}
    
    \begin{Verbatim}[commandchars=\\\{\}]
{\color{incolor}In [{\color{incolor}11}]:} \PY{n}{aggregators} \PY{o}{=} \PY{p}{\PYZob{}}\PY{l+s+s1}{\PYZsq{}}\PY{l+s+s1}{binaryCorrect}\PY{l+s+s1}{\PYZsq{}}\PY{p}{:}\PY{n}{mean}\PY{p}{\PYZcb{}}
         
         \PY{n}{gp} \PY{o}{=} \PY{n}{df}\PY{p}{[}\PY{n}{df}\PY{o}{.}\PY{n}{type}\PY{o}{==}\PY{l+s+s1}{\PYZsq{}}\PY{l+s+s1}{inner\PYZhy{}outer}\PY{l+s+s1}{\PYZsq{}}\PY{p}{]}\PY{o}{.}\PY{n}{groupby}\PY{p}{(}\PY{p}{[}\PY{l+s+s1}{\PYZsq{}}\PY{l+s+s1}{ID}\PY{l+s+s1}{\PYZsq{}}\PY{p}{,} \PY{l+s+s1}{\PYZsq{}}\PY{l+s+s1}{condition}\PY{l+s+s1}{\PYZsq{}}\PY{p}{]}\PY{p}{,} \PY{n}{as\PYZus{}index}\PY{o}{=}\PY{n+nb+bp}{False}\PY{p}{)}\PY{o}{.}\PY{n}{agg}\PY{p}{(}\PY{n}{aggregators}\PY{p}{)}
\end{Verbatim}


    \begin{Verbatim}[commandchars=\\\{\}]
{\color{incolor}In [{\color{incolor}19}]:} \PY{n}{fig}\PY{p}{,} \PY{n}{ax} \PY{o}{=} \PY{n}{subplots}\PY{p}{(}\PY{p}{)}
         
         \PY{n}{sns}\PY{o}{.}\PY{n}{barplot}\PY{p}{(}\PY{n}{x}\PY{o}{=}\PY{l+s+s1}{\PYZsq{}}\PY{l+s+s1}{condition}\PY{l+s+s1}{\PYZsq{}}\PY{p}{,} \PY{n}{y}\PY{o}{=}\PY{l+s+s1}{\PYZsq{}}\PY{l+s+s1}{binaryCorrect}\PY{l+s+s1}{\PYZsq{}}\PY{p}{,} \PY{n}{data}\PY{o}{=}\PY{n}{gp}\PY{p}{,} \PY{n}{ax}\PY{o}{=}\PY{n}{ax}\PY{p}{,} \PY{n}{alpha}\PY{o}{=}\PY{l+m+mi}{0}\PY{p}{,} \PY{n}{capsize}\PY{o}{=}\PY{l+m+mf}{0.2}\PY{p}{,} \PY{n}{errcolor}\PY{o}{=}\PY{l+s+s1}{\PYZsq{}}\PY{l+s+s1}{darkgrey}\PY{l+s+s1}{\PYZsq{}}\PY{p}{)}
         \PY{n}{sns}\PY{o}{.}\PY{n}{swarmplot}\PY{p}{(}\PY{n}{x}\PY{o}{=}\PY{l+s+s1}{\PYZsq{}}\PY{l+s+s1}{condition}\PY{l+s+s1}{\PYZsq{}}\PY{p}{,} \PY{n}{y}\PY{o}{=}\PY{l+s+s1}{\PYZsq{}}\PY{l+s+s1}{binaryCorrect}\PY{l+s+s1}{\PYZsq{}}\PY{p}{,} \PY{n}{data}\PY{o}{=}\PY{n}{gp}\PY{p}{,} \PY{n}{ax}\PY{o}{=}\PY{n}{ax}\PY{p}{,} \PY{n}{color}\PY{o}{=}\PY{l+s+s1}{\PYZsq{}}\PY{l+s+s1}{green}\PY{l+s+s1}{\PYZsq{}}\PY{p}{)}
         
         \PY{n}{ax}\PY{o}{.}\PY{n}{set}\PY{p}{(}\PY{n}{ylim}\PY{o}{=}\PY{p}{(}\PY{o}{\PYZhy{}}\PY{l+m+mf}{0.05}\PY{p}{,}\PY{l+m+mf}{1.05}\PY{p}{)}\PY{p}{)}
         \PY{n}{ax}\PY{o}{.}\PY{n}{plot}\PY{p}{(}\PY{n+nb}{range}\PY{p}{(}\PY{o}{\PYZhy{}}\PY{l+m+mi}{1}\PY{p}{,}\PY{l+m+mi}{4}\PY{p}{)}\PY{p}{,} \PY{p}{[}\PY{l+m+mf}{0.5}\PY{p}{]}\PY{o}{*}\PY{l+m+mi}{5}\PY{p}{,} \PY{n}{color}\PY{o}{=}\PY{l+s+s1}{\PYZsq{}}\PY{l+s+s1}{grey}\PY{l+s+s1}{\PYZsq{}}\PY{p}{,} \PY{n}{linestyle}\PY{o}{=}\PY{l+s+s1}{\PYZsq{}}\PY{l+s+s1}{\PYZhy{}\PYZhy{}}\PY{l+s+s1}{\PYZsq{}}\PY{p}{,} \PY{n}{linewidth}\PY{o}{=}\PY{l+m+mf}{0.5}\PY{p}{)}
\end{Verbatim}


\begin{Verbatim}[commandchars=\\\{\}]
{\color{outcolor}Out[{\color{outcolor}19}]:} [<matplotlib.lines.Line2D at 0x1a19c48f50>]
\end{Verbatim}
            
    \begin{center}
    \adjustimage{max size={0.9\linewidth}{0.9\paperheight}}{output_21_1.png}
    \end{center}
    { \hspace*{\fill} \\}
    
    What are the participants below chance level doing? Particularly in the
AdjNum condition, where group performance is not different from chance
(see stats below)...

What are they answering? Are they selecting the incorrect prenominal
order or are they not doing the task correctly?

    \begin{Verbatim}[commandchars=\\\{\}]
{\color{incolor}In [{\color{incolor}20}]:} \PY{k}{def} \PY{n+nf}{isIncorrectResponseTypePostnominal}\PY{p}{(}\PY{n}{i}\PY{p}{)}\PY{p}{:}
             \PY{n}{row} \PY{o}{=} \PY{n}{df}\PY{o}{.}\PY{n}{loc}\PY{p}{[}\PY{n}{i}\PY{p}{]}
             \PY{k}{if} \PY{n}{row}\PY{o}{.}\PY{n}{binaryCorrect} \PY{o}{==} \PY{l+m+mi}{1}\PY{p}{:}
                 \PY{k}{return} \PY{n}{nan}
             
             \PY{n}{response} \PY{o}{=} \PY{n}{row}\PY{o}{.}\PY{n}{choice}
             \PY{n}{responseType} \PY{o}{=} \PY{n}{row}\PY{p}{[}\PY{l+s+s1}{\PYZsq{}}\PY{l+s+s1}{choiceType}\PY{l+s+s1}{\PYZsq{}}\PY{o}{+}\PY{n+nb}{str}\PY{p}{(}\PY{n}{response}\PY{p}{)}\PY{p}{]}
             
             \PY{k}{try}\PY{p}{:}
                 \PY{k}{if} \PY{n}{responseType}\PY{p}{[}\PY{o}{\PYZhy{}}\PY{l+m+mi}{1}\PY{p}{]} \PY{o}{==} \PY{l+s+s1}{\PYZsq{}}\PY{l+s+s1}{N}\PY{l+s+s1}{\PYZsq{}}\PY{p}{:}
                     \PY{k}{return} \PY{l+m+mi}{0}
                 \PY{k}{else}\PY{p}{:}
                     \PY{k}{return} \PY{l+m+mi}{1}
             \PY{k}{except}\PY{p}{:}
                 \PY{k}{return} \PY{n}{nan}
             
         \PY{n}{df}\PY{p}{[}\PY{l+s+s1}{\PYZsq{}}\PY{l+s+s1}{responsePostnominal}\PY{l+s+s1}{\PYZsq{}}\PY{p}{]} \PY{o}{=} \PY{n}{df}\PY{o}{.}\PY{n}{index}\PY{o}{.}\PY{n}{map}\PY{p}{(}\PY{k}{lambda} \PY{n}{x}\PY{p}{:} \PY{n}{isIncorrectResponseTypePostnominal}\PY{p}{(}\PY{n}{x}\PY{p}{)}\PY{p}{)}
         
         \PY{k}{print} \PY{n}{df}\PY{o}{.}\PY{n}{groupby}\PY{p}{(}\PY{l+s+s1}{\PYZsq{}}\PY{l+s+s1}{condition}\PY{l+s+s1}{\PYZsq{}}\PY{p}{)}\PY{o}{.}\PY{n}{responsePostnominal}\PY{o}{.}\PY{n}{count}\PY{p}{(}\PY{p}{)}
         
         \PY{n}{aggregators} \PY{o}{=} \PY{p}{\PYZob{}}\PY{l+s+s1}{\PYZsq{}}\PY{l+s+s1}{responsePostnominal}\PY{l+s+s1}{\PYZsq{}}\PY{p}{:}\PY{n}{mean}\PY{p}{\PYZcb{}}
         
         \PY{n}{gp} \PY{o}{=} \PY{n}{df}\PY{p}{[}\PY{n}{df}\PY{o}{.}\PY{n}{type}\PY{o}{==}\PY{l+s+s1}{\PYZsq{}}\PY{l+s+s1}{inner\PYZhy{}outer}\PY{l+s+s1}{\PYZsq{}}\PY{p}{]}\PY{o}{.}\PY{n}{groupby}\PY{p}{(}\PY{p}{[}\PY{l+s+s1}{\PYZsq{}}\PY{l+s+s1}{ID}\PY{l+s+s1}{\PYZsq{}}\PY{p}{,} \PY{l+s+s1}{\PYZsq{}}\PY{l+s+s1}{condition}\PY{l+s+s1}{\PYZsq{}}\PY{p}{]}\PY{p}{,} \PY{n}{as\PYZus{}index}\PY{o}{=}\PY{n+nb+bp}{False}\PY{p}{)}\PY{o}{.}\PY{n}{agg}\PY{p}{(}\PY{n}{aggregators}\PY{p}{)}
\end{Verbatim}


    \begin{Verbatim}[commandchars=\\\{\}]
condition
condition\_AdjDem    166
condition\_AdjNum    192
condition\_NumDem    113
Name: responsePostnominal, dtype: int64

    \end{Verbatim}

    It appears that very few responses overall are postnominal, so mostly,
participants are on task. Below is the proportion of non prenominal
isomorphic responses that are actually postnominal. Apparently
participants who are giving non-isomorphic responses are still choosing
the prenominal positioning.

    \begin{Verbatim}[commandchars=\\\{\}]
{\color{incolor}In [{\color{incolor}21}]:} \PY{n}{fig}\PY{p}{,} \PY{n}{ax} \PY{o}{=} \PY{n}{subplots}\PY{p}{(}\PY{p}{)}
         
         \PY{n}{sns}\PY{o}{.}\PY{n}{barplot}\PY{p}{(}\PY{n}{x}\PY{o}{=}\PY{l+s+s1}{\PYZsq{}}\PY{l+s+s1}{condition}\PY{l+s+s1}{\PYZsq{}}\PY{p}{,} \PY{n}{y}\PY{o}{=}\PY{l+s+s1}{\PYZsq{}}\PY{l+s+s1}{responsePostnominal}\PY{l+s+s1}{\PYZsq{}}\PY{p}{,} \PY{n}{data}\PY{o}{=}\PY{n}{gp}\PY{p}{,} \PY{n}{ax}\PY{o}{=}\PY{n}{ax}\PY{p}{,} \PY{n}{alpha}\PY{o}{=}\PY{l+m+mi}{0}\PY{p}{,} \PY{n}{capsize}\PY{o}{=}\PY{l+m+mf}{0.2}\PY{p}{,} \PY{n}{errcolor}\PY{o}{=}\PY{l+s+s1}{\PYZsq{}}\PY{l+s+s1}{darkgrey}\PY{l+s+s1}{\PYZsq{}}\PY{p}{)}
         \PY{n}{sns}\PY{o}{.}\PY{n}{swarmplot}\PY{p}{(}\PY{n}{x}\PY{o}{=}\PY{l+s+s1}{\PYZsq{}}\PY{l+s+s1}{condition}\PY{l+s+s1}{\PYZsq{}}\PY{p}{,} \PY{n}{y}\PY{o}{=}\PY{l+s+s1}{\PYZsq{}}\PY{l+s+s1}{responsePostnominal}\PY{l+s+s1}{\PYZsq{}}\PY{p}{,} \PY{n}{data}\PY{o}{=}\PY{n}{gp}\PY{p}{,} \PY{n}{ax}\PY{o}{=}\PY{n}{ax}\PY{p}{,} \PY{n}{color}\PY{o}{=}\PY{l+s+s1}{\PYZsq{}}\PY{l+s+s1}{green}\PY{l+s+s1}{\PYZsq{}}\PY{p}{)}
         
         \PY{n}{ax}\PY{o}{.}\PY{n}{set}\PY{p}{(}\PY{n}{ylim}\PY{o}{=}\PY{p}{(}\PY{o}{\PYZhy{}}\PY{l+m+mf}{0.05}\PY{p}{,}\PY{l+m+mf}{1.05}\PY{p}{)}\PY{p}{)}
         \PY{n}{ax}\PY{o}{.}\PY{n}{plot}\PY{p}{(}\PY{n+nb}{range}\PY{p}{(}\PY{o}{\PYZhy{}}\PY{l+m+mi}{1}\PY{p}{,}\PY{l+m+mi}{4}\PY{p}{)}\PY{p}{,} \PY{p}{[}\PY{l+m+mf}{0.5}\PY{p}{]}\PY{o}{*}\PY{l+m+mi}{5}\PY{p}{,} \PY{n}{color}\PY{o}{=}\PY{l+s+s1}{\PYZsq{}}\PY{l+s+s1}{grey}\PY{l+s+s1}{\PYZsq{}}\PY{p}{,} \PY{n}{linestyle}\PY{o}{=}\PY{l+s+s1}{\PYZsq{}}\PY{l+s+s1}{\PYZhy{}\PYZhy{}}\PY{l+s+s1}{\PYZsq{}}\PY{p}{,} \PY{n}{linewidth}\PY{o}{=}\PY{l+m+mf}{0.5}\PY{p}{)}
         
         \PY{k}{print} \PY{n}{gp}\PY{o}{.}\PY{n}{groupby}\PY{p}{(}\PY{l+s+s1}{\PYZsq{}}\PY{l+s+s1}{condition}\PY{l+s+s1}{\PYZsq{}}\PY{p}{)}\PY{o}{.}\PY{n}{responsePostnominal}\PY{o}{.}\PY{n}{mean}\PY{p}{(}\PY{p}{)}
\end{Verbatim}


    \begin{Verbatim}[commandchars=\\\{\}]
condition
condition\_AdjDem    0.404880
condition\_AdjNum    0.274049
condition\_NumDem    0.253311
Name: responsePostnominal, dtype: float64

    \end{Verbatim}

    \begin{center}
    \adjustimage{max size={0.9\linewidth}{0.9\paperheight}}{output_25_1.png}
    \end{center}
    { \hspace*{\fill} \\}
    
    Of the low number of responses that are incorrect, most participants are
choosing the non-isomorphic prenominal order. While this goes against
the universal, it at least means that they have understood that
modifiers precede rather than follow the noun in the experiment (i.e.,
they are on-task).

    \subsection{R modelling}\label{r-modelling}

Prepare to do R modelling.

    \begin{Verbatim}[commandchars=\\\{\}]
{\color{incolor}In [{\color{incolor}22}]:} \PY{o}{\PYZpc{}}\PY{k}{load\PYZus{}ext} rpy2.ipython
\end{Verbatim}


    Load in data and create subsets on which each model will be run.

    \begin{Verbatim}[commandchars=\\\{\}]
{\color{incolor}In [{\color{incolor}23}]:} \PY{o}{\PYZpc{}\PYZpc{}}R
         \PY{k+kn}{library}\PY{p}{(}\PY{l+s}{\PYZsq{}}\PY{l+s}{lme4\PYZsq{}}\PY{p}{)}
         
         d \PY{o}{\PYZlt{}\PYZhy{}} read.csv\PY{p}{(}\PY{l+s}{\PYZsq{}}\PY{l+s}{../data/U20Thai\PYZus{}TestData.csv\PYZsq{}}\PY{p}{)}
         test \PY{o}{\PYZlt{}\PYZhy{}} \PY{k+kp}{subset}\PY{p}{(}d\PY{p}{,} d\PY{o}{\PYZdl{}}type\PY{o}{==}\PY{l+s}{\PYZsq{}}\PY{l+s}{inner\PYZhy{}outer\PYZsq{}}\PY{p}{)} \PY{c+c1}{\PYZsh{} only select two modifier trials}
         
         testAdjDem \PY{o}{\PYZlt{}\PYZhy{}} \PY{k+kp}{subset}\PY{p}{(}test\PY{p}{,} test\PY{o}{\PYZdl{}}condition\PY{o}{==}\PY{l+s}{\PYZdq{}}\PY{l+s}{condition\PYZus{}AdjDem\PYZdq{}}\PY{p}{)}
         testAdjNum \PY{o}{\PYZlt{}\PYZhy{}} \PY{k+kp}{subset}\PY{p}{(}test\PY{p}{,} test\PY{o}{\PYZdl{}}condition\PY{o}{==}\PY{l+s}{\PYZdq{}}\PY{l+s}{condition\PYZus{}AdjNum\PYZdq{}}\PY{p}{)}
         testNumDem \PY{o}{\PYZlt{}\PYZhy{}} \PY{k+kp}{subset}\PY{p}{(}test\PY{p}{,} test\PY{o}{\PYZdl{}}condition\PY{o}{==}\PY{l+s}{\PYZdq{}}\PY{l+s}{condition\PYZus{}NumDem\PYZdq{}}\PY{p}{)}
\end{Verbatim}


    \begin{Verbatim}[commandchars=\\\{\}]
/Users/Alexander/anaconda/envs/rpy-env/lib/python2.7/site-packages/rpy2/rinterface/\_\_init\_\_.py:186: RRuntimeWarning: Le chargement a n�cessit� le package : Matrix

  warnings.warn(x, RRuntimeWarning)

    \end{Verbatim}

    \subsubsection{Adj-Dem}\label{adj-dem}

Run models for the adj-dem condition, and test whether intercept is
significantly different from chance level.

    \begin{Verbatim}[commandchars=\\\{\}]
{\color{incolor}In [{\color{incolor}24}]:} \PY{o}{\PYZpc{}\PYZpc{}}R
         
         m1AdjDem \PY{o}{\PYZlt{}\PYZhy{}} glmer\PY{p}{(}binaryCorrect \PY{o}{\PYZti{}} \PY{l+m}{1} \PY{o}{+} \PY{p}{(}\PY{l+m}{1}\PY{o}{|}ID\PY{p}{)} \PY{o}{+} \PY{p}{(}\PY{l+m}{1}\PY{o}{|}mod\PY{p}{)}\PY{p}{,} family\PY{o}{=}binomial\PY{p}{,} data\PY{o}{=}testAdjDem\PY{p}{)}
         m0AdjDem \PY{o}{\PYZlt{}\PYZhy{}} glmer\PY{p}{(}binaryCorrect \PY{o}{\PYZti{}} \PY{l+m}{0} \PY{o}{+} \PY{p}{(}\PY{l+m}{1}\PY{o}{|}ID\PY{p}{)} \PY{o}{+} \PY{p}{(}\PY{l+m}{1}\PY{o}{|}mod\PY{p}{)}\PY{p}{,} family\PY{o}{=}binomial\PY{p}{,} data\PY{o}{=}testAdjDem\PY{p}{)}
         
         anova\PY{p}{(}m1AdjDem\PY{p}{,} m0AdjDem\PY{p}{)}
\end{Verbatim}


    
    \begin{verbatim}
Data: testAdjDem
Models:
m0AdjDem: binaryCorrect ~ 0 + (1 | ID) + (1 | mod)
m1AdjDem: binaryCorrect ~ 1 + (1 | ID) + (1 | mod)
         Df    AIC    BIC  logLik deviance  Chisq Chi Df Pr(>Chisq)   
m0AdjDem  2 673.48 682.28 -334.74   669.48                            
m1AdjDem  3 668.06 681.25 -331.03   662.06 7.4271      1   0.006425 **
---
Signif. codes:  0 ‘***’ 0.001 ‘**’ 0.01 ‘*’ 0.05 ‘.’ 0.1 ‘ ’ 1

    \end{verbatim}

    
    \subsubsection{Adj-Num}\label{adj-num}

Run models for the adj-num condition, and test whether intercept is
significantly different from chance level.

    \begin{Verbatim}[commandchars=\\\{\}]
{\color{incolor}In [{\color{incolor}25}]:} \PY{o}{\PYZpc{}\PYZpc{}}R
         
         m1AdjNum \PY{o}{\PYZlt{}\PYZhy{}} glmer\PY{p}{(}binaryCorrect \PY{o}{\PYZti{}} \PY{l+m}{1} \PY{o}{+} \PY{p}{(}\PY{l+m}{1}\PY{o}{|}ID\PY{p}{)} \PY{o}{+} \PY{p}{(}\PY{l+m}{1}\PY{o}{|}mod\PY{p}{)}\PY{p}{,} family\PY{o}{=}binomial\PY{p}{,} data\PY{o}{=}testAdjNum\PY{p}{)}
         m0AdjNum \PY{o}{\PYZlt{}\PYZhy{}} glmer\PY{p}{(}binaryCorrect \PY{o}{\PYZti{}} \PY{l+m}{0} \PY{o}{+} \PY{p}{(}\PY{l+m}{1}\PY{o}{|}ID\PY{p}{)} \PY{o}{+} \PY{p}{(}\PY{l+m}{1}\PY{o}{|}mod\PY{p}{)}\PY{p}{,} family\PY{o}{=}binomial\PY{p}{,} data\PY{o}{=}testAdjNum\PY{p}{)}
         
         anova\PY{p}{(}m1AdjNum\PY{p}{,} m0AdjNum\PY{p}{)}
\end{Verbatim}


    
    \begin{verbatim}
Data: testAdjNum
Models:
m0AdjNum: binaryCorrect ~ 0 + (1 | ID) + (1 | mod)
m1AdjNum: binaryCorrect ~ 1 + (1 | ID) + (1 | mod)
         Df    AIC    BIC  logLik deviance  Chisq Chi Df Pr(>Chisq)
m0AdjNum  2 598.27 607.06 -297.13   594.27                         
m1AdjNum  3 599.06 612.25 -296.53   593.06 1.2151      1     0.2703

    \end{verbatim}

    
    \subsubsection{Num-Dem}\label{num-dem}

Run models for the num-dem condition, and test whether intercept is
significantly different from chance level.

    \begin{Verbatim}[commandchars=\\\{\}]
{\color{incolor}In [{\color{incolor}26}]:} \PY{o}{\PYZpc{}\PYZpc{}}R
         
         m1NumDem \PY{o}{\PYZlt{}\PYZhy{}} glmer\PY{p}{(}binaryCorrect \PY{o}{\PYZti{}} \PY{l+m}{1} \PY{o}{+} \PY{p}{(}\PY{l+m}{1}\PY{o}{|}ID\PY{p}{)} \PY{o}{+} \PY{p}{(}\PY{l+m}{1}\PY{o}{|}mod\PY{p}{)}\PY{p}{,} family\PY{o}{=}binomial\PY{p}{,} data\PY{o}{=}testNumDem\PY{p}{)}
         m0NumDem \PY{o}{\PYZlt{}\PYZhy{}} glmer\PY{p}{(}binaryCorrect \PY{o}{\PYZti{}} \PY{l+m}{0} \PY{o}{+} \PY{p}{(}\PY{l+m}{1}\PY{o}{|}ID\PY{p}{)} \PY{o}{+} \PY{p}{(}\PY{l+m}{1}\PY{o}{|}mod\PY{p}{)}\PY{p}{,} family\PY{o}{=}binomial\PY{p}{,} data\PY{o}{=}testNumDem\PY{p}{)}
         
         anova\PY{p}{(}m1NumDem\PY{p}{,} m0NumDem\PY{p}{)}
\end{Verbatim}


    
    \begin{verbatim}
Data: testNumDem
Models:
m0NumDem: binaryCorrect ~ 0 + (1 | ID) + (1 | mod)
m1NumDem: binaryCorrect ~ 1 + (1 | ID) + (1 | mod)
         Df    AIC    BIC  logLik deviance  Chisq Chi Df Pr(>Chisq)  
m0NumDem  2 347.92 356.27 -171.96   343.92                           
m1NumDem  3 343.89 356.41 -168.94   337.89 6.0372      1    0.01401 *
---
Signif. codes:  0 ‘***’ 0.001 ‘**’ 0.01 ‘*’ 0.05 ‘.’ 0.1 ‘ ’ 1

    \end{verbatim}

    
    \subsection{Remarks}\label{remarks}

Something funny is going on in the AdjNum condition. Participants seem
less likely to choose the prenominal isomorphic order and are instead
preferring the prenominal non-isomorphic order: Adj Num N. This does not
follow from Gump's original comments suggesting that poor performance
might be to do with their choosing N Num Adj as the order. They do not
seem to be doing that.

Currently seeking to consult Thai speakers about our stimuli before
heading to Bangkok.


    % Add a bibliography block to the postdoc
    
    
    
    \end{document}
